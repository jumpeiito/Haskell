\documentclass[b5j,10pt]{jarticle}
\usepackage{graphics,palatino,multirow}
\pagestyle{empty}
\setlength{\textheight}{120mm}
\setlength{\textwidth}{78mm}
\topmargin -15mm
\headheight 0mm
\headsep 0mm
\oddsidemargin -15mm
\newcommand{\todayd}{\the\year 年\the\month 月\the\day 日}
\newcommand{\dayfigure}{\the\month/\tempDay}
\newcommand{\dayfigureW}{\the\month 月\tempDay 日(\tempDW)}
\newcommand{\rangeMonth}{\newcount\mon \newcount\oldmon \newcount\oldmontemp
  \mon = \the\month
  \oldmon = \the\month
  \advance\oldmon by 10
  \oldmontemp = \the\oldmon
  \divide\oldmontemp by 12
  \multiply\oldmontemp by 12
  \advance\oldmon by -\oldmontemp
  \the\oldmon 〜 \the\mon 月
}
\newcommand{\Joseki}[3]{\begin{center}
  {\Huge \underline{保~~険~~料~~督~~促~~状}}
\end{center}\par\vspace*{.6zw}

\underline {\makebox[7zw]{#1}殿}\par
貴殿は建築国保料を下記の通り滞納しています。

\begin{center}
  {\bf\large
    \dayfigureW 17時までに\par
  醍醐支部事務所へ入金がないと、\par
  保険資格を喪失(除籍)してしまいます。}
\end{center}\par\vspace*{.6zw}

17時までに支部事務所まで届けて下さい。振込や遅れる場合も、必ず支部事務
所までご連絡下さい。\par

連絡なき場合・入金が確認できない場合は、資格を喪失します。いったん除籍
処分をうけると、再加入が困難になります。\par

【なお、行き違いの場合はご容赦下さい。】\par\vspace*{.4zw}
\noindent
\begin{tabular}{|c|c|c|}\hline
  \multirow{2}{*}{滞納月}&\multirow{2}{*}{滞納金額}&\dayfigure までに、少なくとも\\
  &&1ヶ月分#3 円の\\\cline{1-2}
  \rangeMonth&#2 円&入金がないと除籍になります。\\\hline
\end{tabular}\par\vspace*{.5zw}
\todayd\par\vspace*{.3zw}
\begin{flushright}
  京都市伏見区醍醐上ノ山町18-13\par
  全京都建築労働組合 醍醐支部\par
  075-572-4949\par
\end{flushright}}
\newcommand{\tempDay}{}
\newcommand{\tempDW}{}
\begin{document}
\documentclass[b5j,10pt]{jarticle}
\usepackage{graphics,palatino,multirow}
\pagestyle{empty}
\setlength{\textheight}{120mm}
\setlength{\textwidth}{78mm}
\topmargin -15mm
\headheight 0mm
\headsep 0mm
\oddsidemargin -15mm
\newcommand{\todayd}{\the\year 年\the\month 月\the\day 日}
\newcommand{\dayfigure}{\the\month/\tempDay}
\newcommand{\dayfigureW}{\the\month 月\tempDay 日(\tempDW)}
\newcommand{\rangeMonth}{\newcount\mon \newcount\oldmon \newcount\oldmontemp
  \mon = \the\month
  \oldmon = \the\month
  \advance\oldmon by 10
  \oldmontemp = \the\oldmon
  \divide\oldmontemp by 12
  \multiply\oldmontemp by 12
  \advance\oldmon by -\oldmontemp
  \the\oldmon 〜 \the\mon 月
}
\newcommand{\Joseki}[3]{\begin{center}
  {\Huge \underline{保~~険~~料~~督~~促~~状}}
\end{center}\par\vspace*{.6zw}

\underline {\makebox[7zw]{#1}殿}\par
貴殿は建築国保料を下記の通り滞納しています。

\begin{center}
  {\bf\large
    \dayfigureW 17時までに\par
  醍醐支部事務所へ入金がないと、\par
  保険資格を喪失(除籍)してしまいます。}
\end{center}\par\vspace*{.6zw}

17時までに支部事務所まで届けて下さい。振込や遅れる場合も、必ず支部事務
所までご連絡下さい。\par

連絡なき場合・入金が確認できない場合は、資格を喪失します。いったん除籍
処分をうけると、再加入が困難になります。\par

【なお、行き違いの場合はご容赦下さい。】\par\vspace*{.4zw}
\noindent
\begin{tabular}{|c|c|c|}\hline
  \multirow{2}{*}{滞納月}&\multirow{2}{*}{滞納金額}&\dayfigure までに、少なくとも\\
  &&1ヶ月分#3 円の\\\cline{1-2}
  \rangeMonth&#2 円&入金がないと除籍になります。\\\hline
\end{tabular}\par\vspace*{.5zw}
\todayd\par\vspace*{.3zw}
\begin{flushright}
  京都市伏見区醍醐上ノ山町18-13\par
  全京都建築労働組合 醍醐支部\par
  075-572-4949\par
\end{flushright}}
\newcommand{\tempDay}{}
\newcommand{\tempDW}{}
\begin{document}
\documentclass[b5j,10pt]{jarticle}
\usepackage{graphics,palatino,multirow}
\pagestyle{empty}
\setlength{\textheight}{120mm}
\setlength{\textwidth}{78mm}
\topmargin -15mm
\headheight 0mm
\headsep 0mm
\oddsidemargin -15mm
\newcommand{\todayd}{\the\year 年\the\month 月\the\day 日}
\newcommand{\dayfigure}{\the\month/\tempDay}
\newcommand{\dayfigureW}{\the\month 月\tempDay 日(\tempDW)}
\newcommand{\rangeMonth}{\newcount\mon \newcount\oldmon \newcount\oldmontemp
  \mon = \the\month
  \oldmon = \the\month
  \advance\oldmon by 10
  \oldmontemp = \the\oldmon
  \divide\oldmontemp by 12
  \multiply\oldmontemp by 12
  \advance\oldmon by -\oldmontemp
  \the\oldmon 〜 \the\mon 月
}
\newcommand{\Joseki}[3]{\begin{center}
  {\Huge \underline{保~~険~~料~~督~~促~~状}}
\end{center}\par\vspace*{.6zw}

\underline {\makebox[7zw]{#1}殿}\par
貴殿は建築国保料を下記の通り滞納しています。

\begin{center}
  {\bf\large
    \dayfigureW 17時までに\par
  醍醐支部事務所へ入金がないと、\par
  保険資格を喪失(除籍)してしまいます。}
\end{center}\par\vspace*{.6zw}

17時までに支部事務所まで届けて下さい。振込や遅れる場合も、必ず支部事務
所までご連絡下さい。\par

連絡なき場合・入金が確認できない場合は、資格を喪失します。いったん除籍
処分をうけると、再加入が困難になります。\par

【なお、行き違いの場合はご容赦下さい。】\par\vspace*{.4zw}
\noindent
\begin{tabular}{|c|c|c|}\hline
  \multirow{2}{*}{滞納月}&\multirow{2}{*}{滞納金額}&\dayfigure までに、少なくとも\\
  &&1ヶ月分#3 円の\\\cline{1-2}
  \rangeMonth&#2 円&入金がないと除籍になります。\\\hline
\end{tabular}\par\vspace*{.5zw}
\todayd\par\vspace*{.3zw}
\begin{flushright}
  京都市伏見区醍醐上ノ山町18-13\par
  全京都建築労働組合 醍醐支部\par
  075-572-4949\par
\end{flushright}}
\newcommand{\tempDay}{}
\newcommand{\tempDW}{}
\begin{document}
\documentclass[b5j,10pt]{jarticle}
\usepackage{graphics,palatino,multirow}
\pagestyle{empty}
\setlength{\textheight}{120mm}
\setlength{\textwidth}{78mm}
\topmargin -15mm
\headheight 0mm
\headsep 0mm
\oddsidemargin -15mm
\newcommand{\todayd}{\the\year 年\the\month 月\the\day 日}
\newcommand{\dayfigure}{\the\month/\tempDay}
\newcommand{\dayfigureW}{\the\month 月\tempDay 日(\tempDW)}
\newcommand{\rangeMonth}{\newcount\mon \newcount\oldmon \newcount\oldmontemp
  \mon = \the\month
  \oldmon = \the\month
  \advance\oldmon by 10
  \oldmontemp = \the\oldmon
  \divide\oldmontemp by 12
  \multiply\oldmontemp by 12
  \advance\oldmon by -\oldmontemp
  \the\oldmon 〜 \the\mon 月
}
\newcommand{\Joseki}[3]{\begin{center}
  {\Huge \underline{保~~険~~料~~督~~促~~状}}
\end{center}\par\vspace*{.6zw}

\underline {\makebox[7zw]{#1}殿}\par
貴殿は建築国保料を下記の通り滞納しています。

\begin{center}
  {\bf\large
    \dayfigureW 17時までに\par
  醍醐支部事務所へ入金がないと、\par
  保険資格を喪失(除籍)してしまいます。}
\end{center}\par\vspace*{.6zw}

17時までに支部事務所まで届けて下さい。振込や遅れる場合も、必ず支部事務
所までご連絡下さい。\par

連絡なき場合・入金が確認できない場合は、資格を喪失します。いったん除籍
処分をうけると、再加入が困難になります。\par

【なお、行き違いの場合はご容赦下さい。】\par\vspace*{.4zw}
\noindent
\begin{tabular}{|c|c|c|}\hline
  \multirow{2}{*}{滞納月}&\multirow{2}{*}{滞納金額}&\dayfigure までに、少なくとも\\
  &&1ヶ月分#3 円の\\\cline{1-2}
  \rangeMonth&#2 円&入金がないと除籍になります。\\\hline
\end{tabular}\par\vspace*{.5zw}
\todayd\par\vspace*{.3zw}
\begin{flushright}
  京都市伏見区醍醐上ノ山町18-13\par
  全京都建築労働組合 醍醐支部\par
  075-572-4949\par
\end{flushright}}
\newcommand{\tempDay}{}
\newcommand{\tempDW}{}
\begin{document}
\input{hagaki.insert}
\end{document}

\end{document}

\end{document}

\end{document}
